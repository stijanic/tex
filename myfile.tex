\documentclass[12pt]{article}
% We have defined the document to be an article using 12 point font.
% Blank lines mean nothing here, in the preamble.
\usepackage[T1]{fontenc}

\begin{document}
% Begin document "environment".
\section{This is a Section}
\subsection{This is a subsection}
This is the body of the subsection.
I 	can move to a new line 	anytime, and I can put in lots
of blanks 	with no 	effect.

\begin{center}
	The text is centered because I have entered the center environment.
	Text remains centered as long as we remain in this environment.
\end{center}
\begin{flushleft}
	Now we are out of the centering environment, and have begun the
	flushleft environment.
\end{flushleft}
\begin{flushright}
	This is another paragraph, but in the flushright environment.
	You will have occasion to use all four paragraph positions.
\end{flushright}
I am back to normal justification. The added space you see between
the above paragraphs is due to entering those environments.	

\noindent This paragraph is not indented.

This precedes center environment.
\begin{center} This line is centered.
	This continues after centering.
\end{center}
This precedes centerline.
\centerline{This line is centered.}
This continues after centering.




Skipping four lines is the same as skipping one line
--- it starts a new paragraph.
\subsection{Here is another subsection}
\section{Here is another section}

A \textit{group} is defined on a set of elements \dots
\\
This is \textbf{boldface}. \\
This is \textit{italic}. \\
This is \textrm{roman}. \\
This is \textsc{small caps}. \\
This is \textsf{sans serif}. \\
This is \textsl{slanted}. \\
This is \texttt{typewriter}. \\
This is \textbf{\textit{bolditalic}}. \\
This is \textbf{\textsc{bold small caps}}. \\
This is {\Large\textbf{heading}}.\\

You can make the text {\large large} or {\Large larger} or
even {\LARGE larger still}. You can also make it {\huge huge}.
You might want to make something {\small small} or
{\footnotesize smaller} or even {\scriptsize smaller still}.
You can make it really {\tiny tiny}.\\
For example, \begin{large} This is large.\end{large}
produces the same result as {\large This is large.}.\\

It is straightforward to underline text — just write \underline{text}.
We can also frame text just by writing \frame{text}.\\

We can give frame some room around the edges by using \fbox {instead}.\\

Here is an example using the quote environment, which was generated by putting before the text and after it:
\begin{quote}
	The construction of the real number system, notably by Dedekind
	cuts, was motivated by the need to fix calculus, which ran into
	trouble due to insufficient rigor in dealing with limits.
\end{quote}

Here is an example that was created by putting before
the text and after it.
\begin{quotation} 
	\bigskip
	``Computers do not dream, any more than they play. We are
	far from certain what dreams are good for, but we know what
	they indicate: a great deal of information processing goes on far
	beneath the surface of man’s purposive behavior, in ways and
	for reasons that are only very indirectly reflected in his overt
	activity.''

	\hfill —-- Alan M. Turing
\end{quotation}

This is a first line. \bigskip

The space you see above is a big skip. \medskip

The space you see just above is a medium skip. \smallskip

The space you now see just above is a small skip.

This is just an ordinary line space.

\begin{verse}
	\textit{Neglect of mathematics works injury to all knowledge, since he
		who is ignorant of it cannot know the other sciences or the
		things of this world. And what is worse, men who are thus
		ignorant are unable to perceive their own ignorance and so do
		not seek a remedy.
	} \hfill --- Roger Bacon
\end{verse}

\begin{description}
	\item [Basic Document Preparation.] Knowing how to setup ...
	\item [Making Tables.] \LaTeX~ provides a means ...
	\item [Bibliography.] Knowing how to create a bibliography ...
	\item [Mathematics.] This is the power of \LaTeX~ and one ...
	\item [Graphics.] This has progressed a great deal in the ...
	\item [Other.] There are a great many things to learn ...
\end{description}

\begin{description}
	\item \textsf{This is how one item in a description list
		environment looks with no optional text at the beginning.}
\end{description}

\begin{itemize}
	\item This is item 1 and our task has just begun.
	before an item have no effect.
	Blank lines
	\item This is item 2 and we shall limit to just this few.
	A blank line within an item does create a new paragraph,
	using the indentation of the itemize environment.
	\begin{itemize}
		\item A second (nested) itemized list changes the bullet
		and indents another level.
	\end{itemize}
\end{itemize}

\begin{enumerate}
	\item This is item 1, and we are having fun.
	\item This is item 2, and it’s time to number anew.
	\begin{enumerate}
		\item Back to item 1, but we are not yet done.
		\item Two is new.
		\begin{enumerate}
			\item One again!
			\item Two (b) or knot 2b?
		\end{enumerate}
	\end{enumerate}
\end{enumerate}

\bigskip
\begin{tabular}{lcr}
	left & center & right \\
	1 & 2
	& 3
\end{tabular}

\bigskip
\begin{tabular}{|l|c|r|} \hline
	-110 & 120 & -130 \\ \hline
	210 & -220 & 230 \\ \hline
\end{tabular}

\bigskip
\begin{tabular}{l|cc|}
	Name & Test 1 & Test 2 \\ \cline{1-1}
	Bob
	& 67
	& 72 \\
	Sue
	& 72
	& 67 \\ \cline{2-3}
\end{tabular}

\bigskip\textsf{
	\begin{tabular}{lp{.3in}l} \\
		\underline{Table 1} &&
		Table 2 \\ \cline{3-3}
		\\
		\begin{tabular}{|lc|} \hline
			Object
			& Symbols used \\ \hline
			variable & lower case \textrm{Roman} \\
			parameter & \textit{Greek}
			\\
			constant & \textsc{upper case} \textrm{Roman} \\
		\end{tabular}
		&& % Begin Table 2
		\begin{tabular}{|rcc|} \hline
			* & 1 & 2 \\ \cline{2-2}
			& 3 & 4 \\ \cline{1-1}\cline{3-3}
		\end{tabular}
	\end{tabular}
} % end sf

\bigskip
\begin{tabular}{|l|l|} \hline
	This amount of text is too long to fit on one line of
	the page. & This is column 2. \\ \hline
\end{tabular}

\bigskip
\begin{tabular}{|l|l|} \hline
	This amount of text is too long to fit on one
	& This is column 2. \\
	line of the page. & \\ \hline
\end{tabular}

\bigskip
\begin{tabular}{|p{2in}|l|} \hline
	This amount of text is too long to fit on one line of the page.
	& This is column 2. \\ \hline
\end{tabular}

\bigskip
\begin{tabular}{|l|l|} \hline
	\parbox{2in}{This amount of text is too long to fit on
		one line of the page.} & This is column 2. \\ \hline
\end{tabular}

\bigskip
\begin{tabular}{|l|l|} \hline
	\parbox[t]{2in}{This amount of text is too long to fit on
		one line of the page.} & This is column 2. \\ \hline
\end{tabular}

\bigskip
\begin{center}
	\begin{tabular}{ll}
		\parbox[t]{3in}{\begin{flushleft}
				This is column 1, and I might want to display something:

				\medskip\centerline{\fbox{How sweet it is.}}\medskip

				This is not the same as

				\medskip\fbox{\centerline{How sweet it is.}}
							\end{flushleft} }
		& \parbox[t]{1in}{\begin{flushleft}\textsf{This is column 2,
					which I have put in sans serif font.}
							\end{flushleft} }
	\end{tabular}
\end{center}

\begin{center}
	\begin{tabular}{l|cc|c}
		& \multicolumn{2}{|c|}{Test number} \\
		\multicolumn{1}{c|}{Student} &
		1
		&
		2
		& Average \\ \hline
		Bill
		&
		67
		&
		72
		&
		70.5 \\
		Charleetah
		&
		72
		&
		67
		&
		70.5 \\ \hline
		& \multicolumn{2}{c}{Taken in class}
		\\ \cline{2-3}
	\end{tabular}
\end{center}

\end{document}
